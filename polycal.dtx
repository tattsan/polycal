% \iffalse
%
%This work is copyright 2014 Tatsuya SATO.
%
%<*driver>
\documentclass{ltxdoc}
\usepackage{hyperref,polynom,polycal}

\begin{document}
    \DocInput{polycal.dtx}
\end{document}
%</driver>
% \fi
%
%^^A
%^^A  Some definitions used for documentation.
%^^A
% \let\packagename\textsf
% \newenvironment{describe}{\trivlist\item[]}{\endtrivlist}
%^^A
%^^A  end of these definitions
%^^A
%
%\newbox\abstractbox
%\setbox\abstractbox=\vbox{
%   \begin{abstract}
%   The \packagename{polycal} package adds calculus macros to \packagename{polynom}
%   package. It can differentiate or integrate polynomials.
%   \end{abstract}}
%
% \title{The \packagename{Polycal} Package}
% \author{Tatsuya SATO}
% \date{2014/01/12\enspace Version 0.01b\\ \box\abstractbox}
% \maketitle
%
% \section{Introduction}
%
% The \packagename{polynom} package manipulates polynomials.
% It supports multiplication, division, getting GCD of polynomials with 
% Euclidean algorithm, factorization with coefficients in rational numbers.
% But it lacks in calculus operation, and substitution commands.
% \packagename{Polycal} package adds these commands to
% \packagename{polynom} package.\bigskip
% 
% For one variable polynomial $P(x)$, substitution $P(a)$ can be obtained as
% a remainder of division with $x-a$. Also dviding with $(x-a)^2$, one can
% obtain $P'(a)(x-a)+P(a)$ as a remainder, and can get $P'(a)$.
% But \packagename{polynom} package  can't deal with multivariate 
% polynomial enough, so sometimes divsion fails. One can't specify main variable. 
% So I implemented direct substitution commands.
% 
% \section{Acknowledgments}
%
% I wish to thank
% Carsten Heinz, and Hendri Adriaens
% for creation and maintainance of \packagename{polynom} package.
% \section{Usage}
% See the manual of \packagename{polynom} package for a basic usage of original
% commands. This manual explains newly added commands.
%
% \subsection{Definition of polynomials}
% First, specify variables you'll use.
% \begin{quote}
%   |\polyset{vars=xab}|
% \end{quote}
% \polyset{vars=xab}
% Define some polynomials using |\polydefine|.
% \begin{macro}{\polydefine}
% \meta{macro}\marg{polynomial}
% \end{macro}
% \begin{quote}
%   |\polydefine\fx{x^2} \polydefine\gx{(a+b)x-ab}|
% \end{quote}
% \polyset{vars=xab}\polydefine\fx{x^2} \polydefine\gx{(a+b)x-ab}
% Polynomials are set to macros |\fx| and |\gx|.
% You can print them using |\polyprint|:
% \begin{quote}
% |\[ f(x)=\polyprint\fx,\quad\quad  g(x)=\polyprint\gx. \]|
% \end{quote}
% \[ f(x)=\polyprint\fx,\quad\quad  g(x)=\polyprint\gx. \]
% Never print |\fx| or |\gx|  directly.
%\subsection{Differentiation}
% You can differentiate polynomials using |\polydiff|.
% \begin{macro}{\polydiff}
% \meta{macro}\marg{variable}\marg{polynomial}
% \end{macro}
% \begin{quote}
%  |\polydiff\Dfx{x}\fx|\medskip\\
%  |\[ \frac{df(x)}{dx}=\polyprint\Dfx. \]|
% \end{quote}
% \polydiff\Dfx{x}\fx
% \[ \frac{df(x)}{dx}=\polyprint\Dfx. \]
% If you define |\x| as |\polydefine\x{x}| \polydefine\x{x},
% you can write this operation as follows.
% \begin{quote}
% |\polydiff\aa{x}{x^2}   \polydiff\bb{x}\fx|\\
% |\polydiff\cc\x{x^2}    \polydiff\dd\x\fx|
% \end{quote}
% \polydiff\aa{x}{x^2} \polydiff\bb{x}\fx \polydiff\cc\x{x^2} \polydiff\dd\x\fx
% |\[\polyprint\aa, \polyprint\bb, \polyprint\cc, \polyprint\dd.\]|
% \[\polyprint\aa, \polyprint\bb, \polyprint\cc, \polyprint\dd.\]
% \let\aa\relax \let\bb\relax \let\cc\relax \let\cc\relax
% You can differentiate with another variable:
% \begin{quote}
% |\polydiff\Dag{a}\gx \polydiff\Dbg{b}\gx|\\
% |\[ \frac{\partial g(x,a,b)}{\partial a} =\polyprint\Dag,\quad|\\
% |   \frac{\partial g(x,a,b)}{\partial b} =\polyprint\Dbg. \]|
% \end{quote}
% \polydiff\Dag{a}\gx \polydiff\Dbg{b}\gx
% \[
% \frac{\partial g(x,a,b)}{\partial a} =\polyprint\Dag,\quad
% \frac{\partial g(x,a,b)}{\partial b} =\polyprint\Dbg.
% \]
% But in this manual, $g(x)$ will be treated as one variable function.
% \begin{quote}
%  |\polydiff\Dgx{x}\gx  \[ \frac{dg(x)}{dx}=\polyprint\Dgx. \]|
% \end{quote}
%  \polydiff\Dgx{x}\gx  \[ \frac{dg(x)}{dx}=\polyprint\Dgx. \]
%\subsection{Integration}
% Now we will calculate area between $y=f(x)$ and $y=g(x)$.
% \begin{quote}
%   |\polysub\hx\gx\fx|\\
%   |\[ h(x)=g(x)-f(x)=\polyprint\hx. \]|
% \end{quote}
%   \polysub\hx\gx\fx
%   \[ h(x)=g(x)-f(x)=\polyprint\hx. \]
% We will integrate this function.
% \begin{macro}{\polyint}
% \meta{macro}\marg{variable}\marg{polynomial}
% \end{macro}
% This macro caclulates indefinite integral of \meta{polynomial} from 
% \meta{variable}$=0$.
%   \begin{quote}
%     |\polyint\Hx{x}\hx \[ H(x)=\int_0^x h(t)dt= \polyprint\Hx. \]|
%   \end{quote}
%     \polyint\Hx{x}\hx \[ H(x)=\int_0^x h(t)dt= \polyprint\Hx. \]
%\subsection{Substitution}
% And definite integral can be obtained by substitution.
% \begin{macro}{\polysubst}
% \meta{macro}\marg{variable}\marg{polynomial-1}\marg{polynomial-2}
% \end{macro}
% This macro substitute \meta{polynomial-1} for \meta{variable} in \meta{polynomial-2}.
%   \begin{quote}
%     |\polysubst\Ha{x}{a}\Hx \polysubst\Hb{x}{b}\Hx  \polysub\S\Hb\Ha |\\
%     |\[ S=\int_a^bh(x)dx=H(b)-H(a)=\polyprint\S. \]|
%   \end{quote}
%     \polysubst\Ha{x}{a}\Hx \polysubst\Hb{x}{b}\Hx   \polysub\S\Hb\Ha
%     \[ S=\int_a^bh(x)dx=H(b)-H(a)=\polyprint\S. \]
% Unfortunately, factorization of multivariate polynomial is not currently supported,
% so this can not be arranged to $S=\displaystyle\frac{(b-a)^3}{6}$ automatically.
% \medskip\\
% Let's try another substitution. If you substitute $b=a^2-2$ in $S$, then
% \begin{quote}
%   |\polysubst\SS{b}{a^2-2}\S| \\
%   |\[ b=a^2-2\ \  \Rightarrow\ \  S=\polyprint\SS=\polyfactorize\SS. \]|
% \end{quote}
%   \polysubst\SS{b}{a^2-2}\S
%   \[ b=a^2-2\ \  \Rightarrow\ \  S=\polyprint\SS=\polyfactorize\SS. \]
% 
% \begin{macro}{\polysubstnum}
% \meta{macro}\marg{variable}\marg{rational number}\marg{polynomial}
% \end{macro}
% To substitute a number, there's another command. Compared with |\polysubst|, 
% it is based on different algorithm, but not so fast. Experimental.
% \begin{macro}{\polysubstsqrt}
% \meta{macro}\marg{variable}\marg{rational number}\marg{1-variable polynomial}
% \end{macro}
% This macro substitute $\sqrt{\meta{rational number}}$ for \meta{variable}.
% Note that polynomial should be one variable. Let substitute $a=\sqrt5$
% in $S=\polyfactorize\SS$.
% \begin{quote}
% |\polysubstsqrt\SSS{a}{5}\SS \[ S=\polyprint\SSS. \]|
% \end{quote}
% \polysubstsqrt\SSS{a}{5}\SS \[ S=\polyprint\SSS. \]
% To eliminate parentheses, set |\polyset{delims={}{}}|.
% \begin{quote}
%   |\[\polyset{delims={}{}}S=\polyprint\SSS. \]|
% \end{quote}
%   \[\polyset{delims={}{}}S=\polyprint\SSS. \]
% \CheckSum{625}
% \StopEventually{}
% \section{Implementation}
%
% Let's start with identification.
%    \begin{macrocode}
%<*package>
\NeedsTeXFormat{LaTeX2e}
\ProvidesPackage{polycal}[2014/01/11 v0.01b (TS)]
\RequirePackage{polynom}
%    \end{macrocode}
% \subsection{The user commands}
% \begin{macro}{\polydefine}
% \meta{macro}\marg{polynomial}
% \begin{describe}
% Transform \meta{polynomial} to internal representation, and set it to \meta{macro}.
% \end{describe}
%    \begin{macrocode}
\newcommand*\polydefine[2]{\pld@GetPoly#1{}{#2}}
%    \end{macrocode}
% \end{macro}
%

% \begin{macro}{\polydiff}
% \meta{macro}\marg{variable}\marg{polynomial}
% \begin{describe}
% Differentiate \meta{polynomial} with respect to \meta{variable}, 
% and  set it to \meta{macro}.
% \end{describe}
%    \begin{macrocode}
\newcommand*\polydiff[1]{%
  \pld@GetPoly{\pld@polya\pld@polyb}%
  {\pld@DiffPoly#1\pld@polya\pld@polyb
    \ignorespaces}}
%    \end{macrocode}
% \end{macro}
%

% \begin{macro}{\polyint}
% \meta{macro}\marg{variable}\marg{polynomial}
% \begin{describe}
% Integrate \meta{polynomial}  from 0, with respect to \meta{variable}.
% \end{describe}
%    \begin{macrocode}
\newcommand*\polyint[1]{%
  \pld@GetPoly{\pld@polya\pld@polyb}%
  {\pld@IntPoly#1\pld@polya\pld@polyb
    \ignorespaces}}
%    \end{macrocode}
% \end{macro}
%

% \begin{macro}{\polysubstnum}
% \meta{macro}\marg{variable}\marg{rational number}\marg{polynomial}
% \begin{describe}
% Substitute \meta{rational number} for \meta{variable} in  \meta{polynomial}.
% \end{describe}
%    \begin{macrocode}
\newcommand*\polysubstnum[1]{%
  \pld@GetPoly{\pld@polya\pld@polyb\pld@polyc}%
  {\pld@SubstNumPoly#1\pld@polya\pld@polyb\pld@polyc
    \ignorespaces}}
%    \end{macrocode}
% \end{macro}
%

% \begin{macro}{\polypower}
% \meta{macro}\marg{variable}\marg{polynomial}\marg{non negative integer}
% \begin{describe}
% Set \meta{non negative integer}th power of \meta{polynomial} to \meta{macro}.
% \end{describe}
%    \begin{macrocode}
\newcommand*\polypower[1]{%
  \pld@GetPoly{\pld@polya}%
  {\pld@PowerPoly#1\pld@polya}}
%    \end{macrocode}
% \end{macro}
%

% \begin{macro}{\polysubst}
% \meta{macro}\marg{variable}\marg{polynomial-1}\marg{polynomial-2}
% \begin{describe}
% Substitute \meta{polynomial-1} for \meta{variable} in  \meta{polynomial-2}.
% \end{describe}
%    \begin{macrocode}
\newcommand*\polysubst[1]{%
  \pld@GetPoly{\pld@polya\pld@polyb\pld@polyc}%
  {\pld@SubstPolyPoly#1\pld@polya\pld@polyb\pld@polyc
    \ignorespaces}}
%    \end{macrocode}
% \end{macro}
%

% \begin{macro}{\polysubstsqrt}
% \meta{macro}\marg{variable}\marg{rational number}\marg{1-variable polynomial}
% \begin{describe}
% Substitute non negative square root of \meta{rational number} for
%  \meta{variable} in \meta{1-variable polynomial}.
% \end{describe}
%    \begin{macrocode}
\newcommand*\polysubstsqrt[1]{%
  \pld@GetPoly{\pld@polya\pld@polyb\pld@polyc}%
  {\pld@SubstSqrtPoly#1\pld@polya\pld@polyb\pld@polyc
    \ignorespaces}}
%    \end{macrocode}
% \end{macro}
%

% \subsection{Internal routines}
% The following three macros gets parameter in internal representation of factors.
%
% \begin{macro}{\@fifthofsix}
% \begin{macro}{\@secondofthree}
% \begin{macro}{\@thirdofthree}
% \begin{describe}
% For example, |\polydefine\X{X}| is equivalent to |\def\X{\pld@R{1}{1}\pld@V{X}{1}}|,
% and |\@fifthofsix\X| gets variable letter |X|.
% \end{describe}
%    \begin{macrocode}
\providecommand\@fifthofsix[6]{#5}
\providecommand\@secondofthree[3]{#2}
\providecommand\@thirdofthree[3]{#3}
%    \end{macrocode}
% \end{macro}
% \end{macro}
% \end{macro}
%

% \subsubsection{Differentiation}
% \begin{macro}{\pld@DiffPoly}
% \meta{macro a}\meta{macro b}\meta{macro c}
% \begin{describe}
% \meta{macro a} gets the derivative
% $\displaystyle\frac{\,d\meta{macro c}\,}{\,d\meta{macro b}\,}$.
% \end{describe}
%    \begin{macrocode}
\def\pld@DiffPoly#1#2#3{%
      \let\polycalc@tempb\@empty%
      \ifx\@empty#3\relax\else
          \expandafter\pld@DiffPoly@\expandafter#2#3+\relax+%
      \fi
      \let#1\polycalc@tempb \pld@Simplify{#1}%
    }
%    \end{macrocode}
% \end{macro}
%

% \begin{macro}{\pld@DiffPoly@}
% \begin{describe}
% This macro cuts away a monomial in the third argument \meta{macro c} of
% |\pld@DiffPoly|, send it to |\pld@DiffMonom|.
% \end{describe}
%    \begin{macrocode}
\def\pld@DiffPoly@#1#2+{%
  \let\polycalc@tempa\@empty
    \ifx\relax#2 \relax
    \else
        \pld@DiffMonom\polycalc@tempa#1{#2}%
        \pld@ExtendPoly\polycalc@tempb{\polycalc@tempa}%
        \expandafter\pld@DiffPoly@\expandafter#1%
\fi}
%    \end{macrocode}
% \end{macro}
%

% \begin{macro}{\pld@DiffMonom}
% \meta{macro a}\meta{macro b}\meta{macro c}
% \begin{describe}
% This macro decrease the degree of the monomial \meta{macro c} by 1
% with respect to variable \meta{macro b} and send it to |\pld@DiffMonom@|.
% \end{describe}
%    \begin{macrocode}
\def\pld@DiffMonom#1#2#3{%
    \let#1\@empty
    \edef\pldc@var@letter{\expandafter\@fifthofsix#2}%
    \def\@tempb{#3}\edef\@tempc{\noexpand\pld@V{\pldc@var@letter}{-1}}%
    \pld@MultiplyPoly\@tempa\@tempb\@tempc%
    \def\pldc@CallMono@##1{%
    \expandafter\pld@DiffMonom@\expandafter #1\expandafter##1%
    \@tempa\relax\@empty\@empty\relax\relax}%
    \expandafter\pldc@CallMono@\pldc@var@letter
}
%    \end{macrocode}
% \end{macro}
%

% \begin{macro}{\pld@DiffMonom@}
% \meta{macro a}\meta{variable letter}\meta{macro b}\marg{content c}\marg{content d}
% \begin{describe}
% This macro cuts away a factor \meta{macro b}\marg{content c}\marg{content d}, 
% of the monomial argument \meta{macro c} of |\pld@DiffMonom|.
% If the factor is a power of \meta{variable letter}, do $x^{n-1}\ \to\ nx^{n-1}$.
% \end{describe}
%    \begin{macrocode}
\def\pld@DiffMonom@#1#2#3#4#5{%
    \ifx\relax#3
      \relax \expandafter\@gobbletwo
    \else
      \ifx #3\pld@V \if#4#2
            \@tempcnta #5 \advance\@tempcnta 1
            \expandafter\pld@Extend\expandafter#1{\expandafter\pld@R\expandafter{\the\@tempcnta}{1}}%
      \fi\fi
      \pld@Extend#1{\noexpand#3{#4}{#5}}%
    \expandafter\pld@DiffMonom@\expandafter#1\expandafter#2%
    \fi}
%    \end{macrocode}
% \end{macro}
%

% \subsubsection{Integration}
% \begin{macro}{\pld@IntPoly}
% \meta{macro a}\meta{macro b}\meta{macro c}
% \begin{describe}
% \meta{macro a} gets the integration
% $\displaystyle\int_0^{\meta{macro b}}\meta{macro c}d\meta{macro b}$.
% \end{describe}
%    \begin{macrocode}
\def\pld@IntPoly#1#2#3{%
      \let\polycalc@tempb\@empty
      \ifx\@empty#3\relax\else
          \expandafter\pld@IntPoly@\expandafter#2#3+\relax+%
      \fi
    \let#1\polycalc@tempb \pld@Simplify{#1}%
    }
%    \end{macrocode}
% \end{macro}
%


% \begin{macro}{\pld@IntPoly@}
% \begin{describe}
% This macro cuts away a monomial in the third argument \meta{macro c} of
% |\pld@IntPoly|, send it to |\pld@IntMonom|.
% \end{describe}
%    \begin{macrocode}
\def\pld@IntPoly@#1#2+{%
  \let\polycalc@tempa\@empty
    \ifx\relax#2 \relax
    \else
        \pld@IntMonom\polycalc@tempa#1{#2}%
        \pld@ExtendPoly\polycalc@tempb{\polycalc@tempa}%
        \expandafter\pld@IntPoly@\expandafter#1%
\fi}
%    \end{macrocode}
% \end{macro}
%

% \begin{macro}{\pld@IntMonom}
% \meta{macro a}\meta{macro b}\meta{macro c}
% \begin{describe}
% This macro increase the degree of the monomial \meta{macro c} by 1
% with respect to variable \meta{macro b} and send it to |\pld@IntMonom@|.
% \end{describe}
%    \begin{macrocode}
\def\pld@IntMonom#1#2#3{%
    \let#1\@empty
    \def\@tempb{#3}\def\@tempc{#2}%
    \pld@MultiplyPoly\@tempa\@tempb\@tempc
    \edef\pldc@var@letter{\expandafter\@fifthofsix#2}%
    \def\pldc@CallMono@##1{%
    \expandafter\pld@IntMonom@\expandafter #1\expandafter##1%
    \@tempa\relax\@empty\@empty\relax\relax}%
    \expandafter\pldc@CallMono@\pldc@var@letter
}
%    \end{macrocode}
% \end{macro}
%

% \begin{macro}{\pld@IntMonom@}
% \meta{macro a}\meta{variable letter}\meta{macro b}\marg{content c}\marg{content d}
% \begin{describe}
% This macro cuts away a factor \meta{macro b}\marg{content c}\marg{content d}, 
% of the monomial argument \meta{macro c} of |\pld@IntMonom|.
% If the factor is a power of \meta{variable letter}, 
% do $\displaystyle x^{n+1}\ \to\ \frac{x^{n+1}}{n+1}$.
% \end{describe}
%    \begin{macrocode}
\def\pld@IntMonom@#1#2#3#4#5{%
    \ifx\relax#3 \relax
      \relax \expandafter\@gobbletwo
    \else
      \ifx #3\pld@V \if#4#2
        \pld@Extend#1{\noexpand\pld@R{1}{#5}}%
      \fi\fi
      \pld@Extend#1{\noexpand#3{#4}{#5}}%
    \expandafter\pld@IntMonom@\expandafter#1\expandafter#2%
    \fi}
%    \end{macrocode}
% \end{macro}
%

% \subsubsection{Substitution of number}
% \begin{macro}{\pld@SubstNumPoly}
% \meta{macro a}\meta{macro b}\meta{macro c}\meta{macro d}
% \begin{describe}
% Substitute rational number \meta{macro c} for variable \meta{macro b} in the
% polynomial \meta{macro d}, and stores it to \meta{macro a}.
% \end{describe}
%    \begin{macrocode}
\def\pld@SubstNumPoly#1#2#3#4{%
      \let\polycalc@tempb\@empty
      \ifx\@empty#4\relax\else
          \expandafter\pld@SubstNumPoly@\expandafter#2\expandafter#3#4+\relax+%
      \fi
    \let#1\polycalc@tempb \pld@Simplify{#1}%
    }
%    \end{macrocode}
% \end{macro}
%

% \begin{macro}{\pld@SubstNumPoly@}
% \begin{describe}
% This macro cuts away a monomial in the fourth argument \meta{macro d} of
% |\pld@SubstNumPoly|, send it to |\pld@SubstNumMonom|.
% \end{describe}
%    \begin{macrocode}
\def\pld@SubstNumPoly@#1#2#3+{%
  \let\polycalc@tempa\@empty
    \ifx\relax#3 \relax
    \else
        \pld@SubstNumMonom\polycalc@tempa#1#2{#3}%
        \pld@ExtendPoly\polycalc@tempb{\polycalc@tempa}%
        \expandafter\pld@SubstNumPoly@\expandafter#1\expandafter#2%
\fi}
%    \end{macrocode}
% \end{macro}
%

% \begin{macro}{\pld@SubstNumMonom}
% \meta{macro a}\meta{macro b}\meta{macro c}\meta{macro d}
% \begin{describe}
% This macro only replace \meta{macro b} with its variable letter,
% and sends four arguments to |\pld@SubstNumMonom@|.
% \end{describe}
%    \begin{macrocode}
\def\pld@SubstNumMonom#1#2#3#4{%
    \let#1\@empty
    \edef\pldc@var@letter{\expandafter\@fifthofsix#2}%
    \def\pldc@CallMono@##1{%
    \expandafter\pld@SubstNumMonom@\expandafter#1\expandafter##1\expandafter#3\expandafter%
    #4\relax\@empty\@empty\relax\relax}%
    \expandafter\pldc@CallMono@\pldc@var@letter
}
%    \end{macrocode}
% \end{macro}
%

% \begin{macro}{\pld@SubstNumMonom@}
%  \meta{macro a}\meta{variable letter}\meta{macro b}\meta{macro c}\marg{contents d}\marg{contents e}
% \begin{describe}
% This macro cuts away a factor \meta{macro c}\marg{content d}\marg{content e}, 
% of the monomial argument \meta{macro d} of |\pld@SubstNumMonom|.
% If the factor contains \meta{variable letter} as variable,
% this macro substitute the rational number \meta{macro b} for that variable.
% \end{describe}
%    \begin{macrocode}
\def\pld@SubstNumMonom@#1#2#3#4#5#6{%
    \ifx\relax#4 \relax \expandafter\@gobbletwo
    \else
      \ifx #4\pld@V
          \if#5#2
             \def\@tempb{\pld@R{-1}{1}}\def\@tempc{#3}%
             \pld@MultiplyPoly\@tempa\@tempb\@tempc
             \edef\@tempb{\noexpand\pld@V{#5}{1}}%
             \pld@ExtendPoly\@tempb\@tempa \pld@Simplify\@tempb
             \def\@tempc{\pld@V{#5}{#6}}%
             \let\pld@stage\maxdimen
             \pld@DividePoly\@tempc\@tempb \pld@Simplify\pld@remainder
             \pld@Extend#1{\pld@remainder}%
          \else
             \pld@Extend#1{\noexpand#4{#5}{#6}}%
          \fi
      \else
          \pld@Extend#1{\noexpand#4{#5}{#6}}%
      \fi
    \expandafter\pld@SubstNumMonom@\expandafter#1\expandafter#2\expandafter#3%
    \fi}
%    \end{macrocode}
% \end{macro}
%

% \subsubsection{Substitution of polynomial}
% \begin{macro}{\pld@SubstPolyPoly}
% \meta{macro a}\meta{macro b}\meta{macro c}\meta{macro d}
% \begin{describe}
% Substitute polynomial \meta{macro c} for variable \meta{macro b} in the
% polynomial \meta{macro d}, and stores it to \meta{macro a}.
% \end{describe}
%    \begin{macrocode}
\def\pld@SubstPolyPoly#1#2#3#4{%
      \let\polycalc@tempb\@empty
      \ifx\@empty#4\relax\else
          \expandafter\pld@SubstPolyPoly@\expandafter#2\expandafter#3#4+\relax+%
      \fi
    \let#1\polycalc@tempb \pld@Simplify{#1}%
    }
%    \end{macrocode}
% \end{macro}
%

% \begin{macro}{\pld@SubstPolyPoly@}
% \begin{describe}
% This macro cuts away a monomial in the fourth argument \meta{macro d} of
% |\pld@SubstPolyPoly|, send it to |\pld@SubstPolyMonom|.
% \end{describe}
%    \begin{macrocode}
\def\pld@SubstPolyPoly@#1#2#3+{%
  \let\polycalc@tempa\@empty
    \ifx\relax#3 \relax
    \else
        \pld@SubstPolyMonom\polycalc@tempa#1#2{#3}%
        \pld@ExtendPoly\polycalc@tempb{\polycalc@tempa}%
        \expandafter\pld@SubstPolyPoly@\expandafter#1\expandafter#2%
\fi}
%    \end{macrocode}
% \end{macro}
%

% \begin{macro}{\pld@SubstPolyMonom}
% \meta{macro a}\meta{macro b}\meta{macro c}\meta{macro d}
% \begin{describe}
% This macro only replace \meta{macro b} with its variable letter,
% and sends four arguments to |\pld@SubstPolyMonom@|.
% \end{describe}
%    \begin{macrocode}
\def\pld@SubstPolyMonom#1#2#3#4{%
    \def#1{\pld@R{1}{1}}%
    \edef\@tempb{\expandafter\@fifthofsix#2}%
    \edef\pldc@var@letter{\expandafter\@fifthofsix#2}%
    \def\pldc@CallMono@##1{%
    \expandafter\pld@SubstPolyMonom@\expandafter#1\expandafter##1\expandafter#3\expandafter%
    #4\relax\@empty\@empty\relax\relax}%
   \expandafter\pldc@CallMono@\pldc@var@letter
}
%    \end{macrocode}
% \end{macro}
%

% \begin{macro}{\pld@SubstPolyMonom@}
%  \meta{macro a}\meta{variable letter}\meta{macro b}\meta{macro c}\marg{contents d}\marg{contents e}
% \begin{describe}
% This macro cuts away a factor \meta{macro c}\marg{content d}\marg{content e}, 
% of the monomial argument \meta{macro d} of |\pld@SubstPolyMonom|.
% If the factor contains \meta{variable letter} as variable,
% this macro substitute the polynomial \meta{macro b} for that variable.
% \end{describe}
%    \begin{macrocode}
\def\pld@SubstPolyMonom@#1#2#3#4#5#6{%
    \ifx\relax#4 \relax \expandafter\@gobbletwo
    \else
      \ifx #4\pld@V
         \if#5#2
             \edef\@tempa{\number#6}%
             \pld@PowerPoly\@tempb#3\@tempa
             \expandafter\def\expandafter\@tempa\expandafter{#1}%
             \pld@MultiplyPoly#1\@tempa\@tempb
          \else
            \expandafter\def\expandafter\@tempa\expandafter{#1}%
            \edef\@tempb{\noexpand#4{#5}{#6}}%
            \pld@MultiplyPoly#1\@tempa\@tempb
          \fi
      \else
          \expandafter\def\expandafter\@tempa\expandafter{#1}%
          \edef\@tempb{\noexpand#4{#5}{#6}}%
          \pld@MultiplyPoly#1\@tempa\@tempb
      \fi
    \expandafter\pld@SubstPolyMonom@\expandafter#1\expandafter#2\expandafter#3%
    \fi}
%    \end{macrocode}
% \end{macro}
%

% \begin{macro}{\pld@PowerPoly}
% \meta{macro a}\meta{macro b}\marg{number a}
% \begin{describe}
% \meta{macro a} gets a \meta{number a}th power of polynomial \meta{macro b}.
% \meta{number a} should be a non negative integer.
% \end{describe}
%    \begin{macrocode}
\def\pld@PowerPoly#1#2#3{%
  \ifnum#3=\z@ \def#1{\pld@R{1}{1}}%
  \else
    \@tempcnta#3\advance\@tempcnta-1
    \edef\@tempa{\the\@tempcnta}%
    \pld@PowerPoly#1#2\@tempa
    \let\@tempb#1
    \pld@MultiplyPoly#1\@tempb#2%
  \fi
}
%    \end{macrocode}
% \end{macro}
%

% \subsubsection{Substitution of square root}
% \begin{macro}{\pld@SubstSqrtPoly}
% \meta{macro a}\meta{macro b}\meta{macro c}\meta{macro d}
% \begin{describe}
% Substitute non negative square root of rational number \meta{macro c}
% for variable \meta{macro b} in the \textbf{one variable} polynomial \meta{macro d},
% and stores it to \meta{macro a}.
% \end{describe}
%    \begin{macrocode}
\def\pld@SubstSqrtPoly#1#2#3#4{%
  \edef\pldc@var@letter{\expandafter\@fifthofsix#2}%
  \edef\pldc@numer@sqrt{\expandafter\@secondofthree#3}%
  \edef\pldc@denom@sqrt{\expandafter\@thirdofthree#3}%
  \edef\pld@polyd{%
    \noexpand\pld@R{1}{1}\noexpand\pld@V{\pldc@var@letter }{2}%
    +\noexpand\pld@R{-\pldc@numer@sqrt }{\pldc@denom@sqrt }}%
  \let\pld@stage\maxdimen
  \pld@DividePoly#4\pld@polyd
  \let\pldc@main@formula\pld@remainder
  \pld@FDefSqrt{\pldc@numer@sqrt}{\pldc@denom@sqrt}\let\pldc@sqrt@inner\pld@temp
  \def\pldc@sqrt@outer{\pld@R {1}{1}\pld@S {\pldc@sqrt@inner }{1}}%
  \pld@SubstNumPoly#1#2\pldc@sqrt@outer\pldc@main@formula%
}
%    \end{macrocode}
% \end{macro}
%

% \subsection{Miscellany}
% \begin{macro}{\pld@IfMonomL@}
% \begin{describe}
% A bug fix of |\pld@IfMonomL@| in polynom.sty.
% \end{describe}
%    \begin{macrocode}
%%%%%% The folowing is a bug fix of \pld@IfMonomL@ in polynom.sty.
\def\pld@IfMonomL@#1\pld@V#2#3#4\@empty#5\pld@V#6#7#8\@empty{%
    \let\pld@next\@empty
          \ifx #6\relax \aftergroup\@secondoftwo
    \else \ifx #2\relax \aftergroup\@firstoftwo
    \else
        \def\pld@va{#2}\def\pld@vb{#6}%
        \ifx\pld@va\pld@vb
            \ifnum#3=#7\relax
                \def\pld@next{\pld@IfMonomL@#4\@empty#8\@empty}%
            \else
                \ifnum#3<#7\relax \aftergroup\@firstoftwo
                            \else \aftergroup\@secondoftwo \fi
            \fi
        \else
            % \pld@IfVarL#2\relax\@empty#6\relax\@empty
            \pld@IfVarL#2#6%
                {\aftergroup\@firstoftwo}%
                {\aftergroup\@secondoftwo}%
        \fi
    \fi \fi
    \pld@next}
%    \end{macrocode}
% \end{macro}
%
%    \begin{macrocode}
%</package>
%    \end{macrocode}
%
%
% \begingroup\small
% \section{History}
% \renewcommand\labelitemi{--}
% \begin{itemize}
% \item[0.01] 2014/01/09 (private test version)
%   \item Implemented |\pld@DiffPoly|,  |\pld@IntPoly|, |\pld@SubstNumPoly|, 
%    |\pld@SubstPlyPoly|, |\pld@SubstSqrtPoly|.
% \item[0.01a] 2014/01/11
%   \item Fixed a bug in |\pld@IfMonomL@| of original \packagename{polynom} package.
% \item[0.01b] 2014/01/12
%   \item Wrote .dtx.
%   \item Minor fix in |\pld@SubstSqrtPoly|.
% \end{itemize}
% \medskip
%
% \noindent TODO:
% \begin{itemize}
%   \item User definable order of variables.
%   \item Rational functions and its derivatives.
%   \item Vector calculus.
%   \item Differential geometric operations.
% \end{itemize}
% \endgroup
%
%
% \Finale
%
%%
%%
\endinput
