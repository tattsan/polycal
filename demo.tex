\documentclass[dvipdfmx]{jsarticle}
\usepackage[T1]{fontenc}
\usepackage{polynom}
\usepackage{polycal2}
\usepackage{amsmath}
\usepackage{moreverb}
\usepackage{pict2e,color}
\usepackage{hyperref}
\usepackage{otf}


\newcommand*{\VAbox}[2][b]{%
\leavevmode\begingroup\setbox0=\hbox{#2}\parbox[#1]{\wd0}{\vspace{0pt}\box0}\endgroup
}


\title{\textsf{polycal}パッケージのデモ}\author{tattsan}\date{}



\begin{document}
\maketitle

\section{動機}
ZR氏の「冬休み課題\ajMaru2:円周率を使わない話」を読んだ。
\begin{quote}
  \url{http://d.hatena.ne.jp/zrbabbler/20131231/1388478052}
  (\url{http://t.co/NjEtKJwLyO})
\end{quote}
\verb+\expandafter+  がよくわからない人のための課題とあり、
自分にはちょうどよいと思った。ただ
\begin{quote}
  「もはや全く以て TeX ネタじゃない…」  
\end{quote}
というZR氏の一人ボケツッコミが気になった。
これは是非とも\textbf{\TeX で解かねばなるまい}。
そこで既存の \textsf{polynom} パッケージに微分や積分のルーチンを追加する
\textsf{polycal} パッケージを作成した。
\begin{quote}
  \url{https://github.com/tattsan/polycal}
\end{quote}
その結果、大して理解していない \verb|\expandafter| をあちこちに書く羽目になった。
以下\textsf{polycal} パッケージを利用してニセ円の面積を計算する。
\section{問題の説明}
\begin{quote}
\begin{boxedverbatim}
%(with pict2e and color package)
\setlength{\unitlength}{1cm}
\begin{picture}(2,2)
\color{blue}\put(1,1){\circle*{2}}
\end{picture}
\end{boxedverbatim}
\hfil
\setlength{\unitlength}{1cm}
  \begin{picture}(2,2)
    \color{blue}\put(1,1){\circle*{2}}
  \end{picture}
\end{quote}
課題は「上の図形の面積を円周率を用いずに求めてください」というもの。
「円周率を用いずに」の部分もZR氏のサイトに解説されている。
\textsf{pict2e}が描く円は3次ベジェ曲線による近似であり、上の例の第1象限部分は
\[
\begin{pmatrix}  x \\ y \end{pmatrix}
= (1-t)^3\begin{pmatrix}  1 \\ 0 \end{pmatrix}
+ 3(1-t)^2t\begin{pmatrix}  1 \\ a \end{pmatrix}
+ 3(1-t)t^2\begin{pmatrix}  a \\ 1 \end{pmatrix}
+ t^3\begin{pmatrix}  0 \\ 1 \end{pmatrix}
\qquad\left(a=\dfrac43\bigl(\sqrt2-1\bigr),\  0\le t\le 1\right)
\]
で表される。このため囲まれた部分の面積はせいぜい2次の無理数で $\pi$ は現れない。

\section{面積の計算}
まず変数の宣言。本当の変数$t$と文字定数$a,b$を変数扱いする。
\polyset{vars=tab}
\begin{quote}
\begin{boxedverbatim}
\polyset{vars=tab}
\end{boxedverbatim}
\end{quote}
次に3次ベジェによる円もどき曲線の定義。
\polydefine\x{(1-t)^3*1+3(1-t)^2t*1+3(1-t)t^2*a+t^3*0}
\polydefine\y{(1-t)^3*0+3(1-t)^2t*a+3(1-t)t^2*1+t^3*1}
\begin{quote}
\begin{boxedverbatim}
\polydefine\x{(1-t)^3*1+3(1-t)^2t*1+3(1-t)t^2*a+t^3*0}
\polydefine\y{(1-t)^3*0+3(1-t)^2t*a+3(1-t)t^2*1+t^3*1}
\end{boxedverbatim}
\end{quote}
ここで $a$ は
\[
  a=\dfrac43\bigl(\sqrt2-1\bigr)
\]
であるが、この値の代入は最後に行なう。
この \verb+\x,\y+ を表示させるには \verb+\polyprint+ を用いる。
\begin{quote}
\begin{boxedverbatim}
\begin{align*}
  x &= \polyprint\x, & y &= \polyprint\y.
\end{align*}    
\end{boxedverbatim}
\end{quote}
\begin{align*}
  x &= \polyprint\x, & y &= \polyprint\y.
\end{align*}
表示されるのは簡約後の数式である。次にこれを微分する。
\polydiff\dx{t}\x  \polydiff\dy{t}\y
\begin{quote}
\begin{boxedverbatim}
\polydiff\dx{t}\x  \polydiff\dy{t}\y
\begin{align*}
  dx &= (\polyprint\dx)\,dt, & dy &= (\polyprint\dy)\,dt.
\end{align*}    
\end{boxedverbatim}
\end{quote}
\begin{align*}
  dx &= (\polyprint\dx)\,dt, & dy &= (\polyprint\dy)\,dt.
\end{align*}    
次に面積要素を定義する。円の面積を計算するため、
$\dfrac12(xdy-ydx)$ の4倍を$dS$と定めよう。
\begin{quote}
\begin{boxedverbatim}
\polymul\dSA\x\dy  \polymul\dSB\y\dx
\polysub\dS\dSA\dSB \polymul\dS{2}\dS
\begin{align*}
 dS &=2(xdy-ydx)\\
    &=(\polyprint\dS)\,dt.
\end{align*}
\end{boxedverbatim}
\end{quote}
\polymul\dSA\x\dy  \polymul\dSB\y\dx
\polysub\dS\dSA\dSB  \polymul\dS{2}\dS
\begin{align*}
 dS &=2(xdy-ydx)\\
    &=(\polyprint\dS)\,dt.
\end{align*}
次は積分だ。
\begin{quote}
\begin{boxedverbatim}
\polyint\St{t}\dS
\polysubstnum\S{t}{1}\St
\[ S = \int_0^1dS = \polyprint\S. \]
\end{boxedverbatim}
\end{quote}
\polyint\St{t}\dS
\polysubstnum\S{t}{1}\St
\[ S = \int_0^1dS = \polyprint\S. \]
最後に$a$の値を代入しよう。まずは$b=\sqrt2$として、$a=\dfrac43(b-1)$を代入する。
\begin{quote}
\begin{boxedverbatim}
\polysubst\S{a}{(4/3)(b-1)}\S
\[ S=\polyprint\S. \]
\end{boxedverbatim}
\end{quote}
\polysubst\S{a}{(4/3)(b-1)}\S
\[ S=\polyprint\S. \]
最後に$b$に$\sqrt2$を代入する。これが答だ!
\begin{quote}
\begin{boxedverbatim}
\polysubstsqrt\S{b}{2}\S \Huge
\[ \polyset{delims={\left.}{\right.}} S=\polyprint\S. \]
\end{boxedverbatim}
\end{quote}
\polysubstsqrt\S{b}{2}\S \Huge
\[ \polyset{delims={\left.}{\right.}} S=\polyprint\S. \]
\normalsize
\section{パラメータ$\boldsymbol a$のチューニング}
ところで $a=\dfrac43\bigl(\sqrt2-1\bigr)$ という値は、曲線が
$t=\dfrac12$ で円上の点 $\left(\dfrac1{\sqrt2\ },\dfrac1{\sqrt2\ }\right)$
を通過するように決めたものらしい。これだとベジェ曲線が完全に円の外側に
位置することになるが、もう少し内側にひっこめた方が円からの最大偏差を減らせる。
そこで例えば
\[
\int_0^1(x^2+y^2)dt=1
\]
となるように$a$を定めてみればどうだろう。これも\TeX で計算してみよう。
\begin{quote}
\begin{boxedverbatim}
\polymul\xx\x\x \polymul\yy\y\y \polyadd\rr\xx\yy
\polyint\It{t}\rr \polysubstnum\I{t}{1}\It
\[ I = \int_0^1(x^2+y^2)dt= \polyprint\I \]
\end{boxedverbatim}
\end{quote}
\polymul\xx\x\x \polymul\yy\y\y \polyadd\rr\xx\yy
\polyint\It{t}\rr \polysubstnum\I{t}{1}\It
\[ I = \int_0^1(x^2+y^2)dt= \polyprint\I \]

\begin{quote}
\begin{boxedverbatim}
\polysub\J\I{1}
\begin{align*}
 I-1 &= \polyprint\J \\ &= \polyfactorize\J
\end{align*}
\end{boxedverbatim}
\end{quote}
\polysub\J\I{1}
\begin{align*}
 I-1 &= \polyprint\J \\ &= \polyfactorize\J
\end{align*}
これより $a=\dfrac1{12}\bigl(\sqrt{385}-13\bigr)$ が得られる。
再度面積を計算すると
\begin{quote}
\begin{boxedverbatim}
\polysubstnum\SS{t}{1}\St
\polysubst\SS{a}{(1/12)(b-13)}\SS
\polysubstsqrt\SS{b}{385}\SS \Huge
\[ \polyset{delims={\left.}{\right.}} S_2=\polyprint\SS. \]
\end{boxedverbatim}
\end{quote}
\polysubstnum\SS{t}{1}\St
\polysubst\SS{a}{(1/12)(b-13)}\SS
\polysubstsqrt\SS{b}{385}\SS \Huge
\[ \polyset{delims={\left.}{\right.}} S_2=\polyprint\SS. \]
\normalsize

\end{document}
%%% Local Variables: 
%%% mode: japanese-latex
%%% TeX-master: t
%%% End: 
