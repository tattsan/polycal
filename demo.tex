\documentclass[dvipdfmx]{jsarticle}
% \documentclass{ltjsarticle}\usepackage[ipaex]{luatexja-preset}
% \documentclass[xelatex,jadriver=standard]{bxjsarticle}
% \documentclass[pdflatex,jadriver=standard]{bxjsarticle}% 因数分解に失敗する。

\usepackage{polycal}% Requires polynom.sty.

\makeatletter
\ifdefined\kanjiskip
   \unless\ifdefined\kchar
      \DeclareKanjiFamily{JY1}{utfj}{}
      \DeclareFontShape{JY1}{utfj}{m}{2}{<->s * [0.92469] utfjmr2-h}{}
      \DeclareFontShape{JY1}{utfj}{bx}{2}{<->s * [0.92469] utfjgr2-h}{}
      \def\maru2{{%
         \edef\reserved@a{\gtdefault}\ifx\k@family\reserved@a \kanjiseries{bx}\fi
         \kanjifamily{utfj}\kanjishape{2}\selectfont \char"C1D1
       }}
   \else
      \def\maru2{②}
   \fi
\else
   \def\maru2{②}
\fi
\makeatother

\usepackage{amsmath}
\usepackage{moreverb}
\usepackage{pict2e,color}
\usepackage{hyperref}
\hypersetup{%
 bookmarksnumbered=true,%
 colorlinks=true,%
 setpagesize=false,%
}
\usepackage{pxjahyper}
\usepackage{xintexpr}
\def\currentTeXversion{3.14159265}
\def\futureTeXversion {3.14159265358979323846264338327950288419716939937510\dots}

\title{\textsf{polycal}パッケージのデモ}
\author{tattsan}
\date{2014/01/12\enspace Version 0.01b}

\begin{document}
\maketitle
\section{動機}
ZR氏の「冬休み課題\maru2:円周率を使わない話」を読んだ。
\begin{quote}
  \url{http://d.hatena.ne.jp/zrbabbler/20131231/1388478052}
  (\url{http://t.co/NjEtKJwLyO})
\end{quote}
\verb+\expandafter+  がよくわからない人のための課題とあり、
自分にはちょうどよいと思った。ただ
\begin{quote}
  「もはや全く以て TeX ネタじゃない…」  
\end{quote}
というZR氏の一人ボケツッコミが気になった。
これは是非とも\textbf{\TeX で解かねばなるまい}。
そこで既存の \textsf{polynom} パッケージに微分や積分のルーチンを追加する
\textsf{polycal} パッケージを作成した。
\begin{quote}
  \url{https://github.com/tattsan/polycal}
\end{quote}
その結果、大して理解していない \verb|\expandafter| をあちこちに書く羽目になった。
以下\textsf{polycal} パッケージを利用してニセ円の面積を計算する。
また\textsf{polynom} パッケージを\textbf{親パッケージ}と呼んで参照する。

\section{問題の説明}
\begin{quote}
\begin{boxedverbatim}
%(with pict2e and color package)
\setlength{\unitlength}{1cm}
\begin{picture}(2,2)
\color{blue}\put(1,1){\circle*{2}}
\end{picture}
\end{boxedverbatim}
\hfil
\setlength{\unitlength}{1cm}
  \begin{picture}(2,2)
    \color{blue}\put(1,1){\circle*{2}}
  \end{picture}
\end{quote}
課題は「上の図形の面積を円周率を用いずに求めてください」というもの。
「円周率を用いずに」の部分もZR氏のサイトに解説されている。
\textsf{pict2e}が描く円は3次ベジェ曲線による近似であり、上の例の第1象限部分は
\[
\begin{pmatrix}  x \\ y \end{pmatrix}
= (1-t)^3\begin{pmatrix}  1 \\ 0 \end{pmatrix}
+ 3(1-t)^2t\begin{pmatrix}  1 \\ a \end{pmatrix}
+ 3(1-t)t^2\begin{pmatrix}  a \\ 1 \end{pmatrix}
+ t^3\begin{pmatrix}  0 \\ 1 \end{pmatrix}
\qquad\left(a=\frac43\bigl(\sqrt2-1\bigr),\  0\le t\le 1\right)
\]
で表される。このため囲まれた部分の面積はせいぜい2次の無理数であり、$\pi$ は現れない。



\section{面積の計算}
まず親パッケージの \verb|\polyset| を用いて変数を宣言する。
ここでは本当の変数$t$だけでなく文字定数$a,b$を変数扱いしておく。
\polyset{vars=tab}
\begin{quote}
\begin{boxedverbatim}
\polyset{vars=tab}
\end{boxedverbatim}
\end{quote}
次に3次ベジェによる円もどき曲線の定義。多項式の定義には \verb|\polydefine| を用いる。
\polydefine\x{(1-t)^3*1+3(1-t)^2t*1+3(1-t)t^2*a+t^3*0}
\polydefine\y{(1-t)^3*0+3(1-t)^2t*a+3(1-t)t^2*1+t^3*1}
\begin{quote}
\begin{boxedverbatim}
\polydefine\x{(1-t)^3*1+3(1-t)^2t*1+3(1-t)t^2*a+t^3*0}
\polydefine\y{(1-t)^3*0+3(1-t)^2t*a+3(1-t)t^2*1+t^3*1}
\end{boxedverbatim}
\end{quote}
これで \verb|\x| や \verb|\y| に多項式がセットされる。
以下でも \verb|\poly***| の形のコマンドの多くは  \verb|\poly***\cs{*}{*}| の形式で
\verb|\cs| に計算結果をセットするようになっている。詳しくはマニュアル \verb|polycal.pdf| を
参照のこと。なお式中の $a$ は
\[
  a=\frac43\bigl(\sqrt2-1\bigr)
\]
だが、値の代入は最後に行なう。
\verb+\x,\y+ の定義内容を表示させるには親パッケージの \verb+\polyprint+ を用いる。
\begin{quote}
\begin{boxedverbatim}
\begin{align*}
  x &= \polyprint\x, & y &= \polyprint\y.
\end{align*}    
\end{boxedverbatim}
\end{quote}
\begin{align*}
  x &= \polyprint\x, & y &= \polyprint\y.
\end{align*}
表示されるのは簡約後の数式である。次にこれを \verb|\polydiff| を用いて微分する。
\polydiff\dx{t}\x  \polydiff\dy{t}\y
\begin{quote}
\begin{boxedverbatim}
\polydiff\dx{t}\x  \polydiff\dy{t}\y
\begin{align*}
  dx &= (\polyprint\dx)\,dt, & dy &= (\polyprint\dy)\,dt.
\end{align*}    
\end{boxedverbatim}
\end{quote}
\begin{align*}
  dx &= (\polyprint\dx)\,dt, & dy &= (\polyprint\dy)\,dt.
\end{align*}    
次に面積要素を定義する。円の面積を計算するため、$\frac12(xdy-ydx)$ の4倍を$dS$と定めよう。
積や差の計算に用いる \verb|\polymul| や \verb|\polysub| は親パッケージのものである。
\begin{quote}
\begin{boxedverbatim}
\polymul\dSA\x\dy  \polymul\dSB\y\dx
\polysub\dS\dSA\dSB \polymul\dS{2}\dS
\begin{align*}
 dS &=2(xdy-ydx)\\
    &=(\polyprint\dS)\,dt.
\end{align*}
\end{boxedverbatim}
\end{quote}
\polymul\dSA\x\dy  \polymul\dSB\y\dx
\polysub\dS\dSA\dSB  \polymul\dS{2}\dS
\begin{align*}
 dS &=2(xdy-ydx)\\
    &=(\polyprint\dS)\,dt.
\end{align*}
次は積分だ。不定積分の計算には \verb|\polyint| を用いる。変数の値が0からの積分になる。
\begin{quote}
\begin{boxedverbatim}
\polyint\St{t}\dS
\[ S(t) = \int_0^tdS = \polyprint\St. \]
\end{boxedverbatim}
\end{quote}
\polyint\St{t}\dS
\[ S(t) = \int_0^tdS = \polyprint\St. \]
定積分を得るには代入コマンド \verb|\polysubstnum| または  \verb|\polysubst| を用いる。
\begin{quote}
\begin{boxedverbatim}
\polysubstnum\S{t}{1}\St
\[ S = \int_0^1dS = S(1) = \polyprint\S. \]
\end{boxedverbatim}
\end{quote}
\polysubstnum\S{t}{1}\St
\[ S = \int_0^1dS = S(1) = \polyprint\S. \]
\verb|\polysubstnum| は有理数限定での代入だが、
\verb|\polysubst| は多項式を(従って有理数も変数も)代入できる。
異なるアルゴリズムを用いてはいるが、現状では \verb|\polysubstnum| が
特に速いというわけではない。
\medskip\\
次に$a$に $\frac43\bigl(\sqrt2-1\bigr)$ を代入しよう。
まずは$b=\sqrt2$として、$a=\frac43(b-1)$を代入する。
\begin{quote}
\begin{boxedverbatim}
\polysubst\S{a}{(4/3)(b-1)}\S
\[ S=\polyprint\S. \]
\end{boxedverbatim}
\end{quote}
\polysubst\S{a}{(4/3)(b-1)}\S
\[ S=\polyprint\S. \]
最後に$b$に$\sqrt2$を代入する。正の有理数の正の平方根の代入には \verb|\polysubstsqrt| を用いる。答はこうだ!
\begin{quote}
\begin{boxedverbatim}
\polysubstsqrt\S{b}{2}\S \Huge
\[ \polyset{delims={\left.}{\right.}} S=\polyprint\S. \]
\end{boxedverbatim}
\end{quote}
\polysubstsqrt\S{b}{2}\S \Huge
\[ \polyset{delims={\left.}{\right.}} S=\polyprint\S. \]
\normalsize
なおこの値を \textsf{xintexpr} パッケージを用いて数値計算してみると次の値になる。
\begin{quote}
\begin{boxedverbatim}
\[ S=\xinttheexpr trunc((-22/5)+(16/3)*sqrt(2,15),10)\relax\dots \]
\end{boxedverbatim}
\end{quote}
\begin{align*}
  S
  &=\xinttheexpr trunc((-22/5)+(16/3)*sqrt(2,15),10)\relax\dots \\
  \text{Current TeX version}
  &=\currentTeXversion \hphantom{\text{current TeX version}}
\end{align*}
誤差はおよそ$+0.028\%$である。

\section{パラメータ\texorpdfstring{$\boldsymbol a$}{a}のチューニング}
ところで $a=\frac43\bigl(\sqrt2-1\bigr)$ という値は、曲線が
$t=\frac12$ で円上の点 $\bigl(\frac1{\sqrt2\,},\frac1{\sqrt2\,}\bigr)$
を通過するように決めたものらしい。これだとベジェ曲線が完全に円の外側に
位置することになるが、もう少し内側にひっこめた方が円からの最大偏差を減らせる。
そこで例えば
\[
\int_0^1(x^2+y^2)dt=1
\]
となるように$a$を定めてみればどうだろう。これも\TeX で計算してみよう。
\begin{quote}
\begin{boxedverbatim}
\polymul\xx\x\x \polymul\yy\y\y \polyadd\rr\xx\yy
\polyint\It{t}\rr \polysubstnum\I{t}{1}\It
\[ I = \int_0^1(x^2+y^2)dt= \polyprint\I. \]
\end{boxedverbatim}
\end{quote}
\polymul\xx\x\x \polymul\yy\y\y \polyadd\rr\xx\yy
\polyint\It{t}\rr \polysubstnum\I{t}{1}\It
\[ I = \int_0^1(x^2+y^2)dt= \polyprint\I. \]

\begin{quote}
\begin{boxedverbatim}
\polysub\J\I{1}
\begin{align*}
 I-1 &= \polyprint\J \\ &= \polyfactorize\J.
\end{align*}
\end{boxedverbatim}
\end{quote}
\verb|\polyfactorize|は親パッケージのコマンドで、
基本的には有理係数の1次因子しか括り出せない。
ただし最後の因子が有理係数の2次式の場合のみ、実係数の範囲で
因数分解してくれる。
\polysub\J\I{1}
\begin{align*}
 I-1 &= \polyprint\J \\ &= \polyfactorize\J.
\end{align*}
これより $a=\frac1{12}\bigl(\sqrt{385}-13\bigr)$ が得られる。
再度面積を計算すると次のようになる。
\begin{quote}
\begin{boxedverbatim}
\polysubstnum\SS{t}{1}\St
\polysubst\SS{a}{(1/12)(b-13)}\SS
\polysubstsqrt\SS{b}{385}\SS \Huge
\[ \polyset{delims={\left.}{\right.}} S_2=\polyprint\SS. \]
\end{boxedverbatim}
\end{quote}
\polysubstnum\SS{t}{1}\St
\polysubst\SS{a}{(1/12)(b-13)}\SS
\polysubstsqrt\SS{b}{385}\SS \Huge
\[ \polyset{delims={\left.}{\right.}} S_2=\polyprint\SS. \]
\normalsize
\begin{quote}
\begin{boxedverbatim}
\[ S_2=\xinttheexpr trunc((-349/120)+(37/120)*sqrt(385,15),10)\relax\dots \]
\end{boxedverbatim}
\end{quote}
\begin{align*}
  S_2
  &=\xinttheexpr trunc((-349/120)+(37/120)*sqrt(385,15),10)\relax\ldots\\
  \text{Future TeX version}
  &=\futureTeXversion
\end{align*}
誤差はおよそ$+0.00035\%$である。
\end{document}
%%% Local Variables: 
%%% mode: japanese-latex
%%% TeX-master: t
%%% End: 
